\documentclass[10pt]{article}

\usepackage{booktabs}

% Formatting
\usepackage[left=0.5in, right=0.5in, top=0.5in, bottom=0.5in]{geometry}
\usepackage{titlesec}
\titleformat*{\section}{\large\bfseries}
\titleformat*{\subsection}{\normalsize\bfseries}
\titleformat*{\subsubsection}{\large\bfseries}

% line spacing
\titlespacing{\section}{0pt}{\parskip}{-\parskip}
\titlespacing{\subsection}{0pt}{\parskip}{-\parskip}
\titlespacing{\subsubsection}{0pt}{\parskip}{-\parskip}

% Math symobls
\usepackage{amsmath, amssymb, mathtools}
%% Abstraction
%%% Expectation
\DeclareMathOperator{\E}{\mathop{{}\mathbb{E}}} % for expectation symbol
\DeclareMathOperator{\sigmoid}{sigmoid}
\DeclareMathOperator{\T}{T}
\DeclareMathOperator{\x}{\mathbf{x}}
\DeclareMathOperator{\Wn}{W_n}

%%% Comment
\newcommand{\comment}[1]{%
  \text{\phantom{(#1)}} \tag{#1}
}
%%% Iverson bracket
\usepackage{stmaryrd}
\usepackage{textcomp}
\usepackage{amsfonts}
\usepackage{amssymb}
%%% Bold 1 for indicator function
\usepackage{dsfont}

\usepackage{xcolor}

\newcommand{\toptitlebar}{
  \hrule height 4pt
  \vskip 0.25in
  \vskip -\parskip%
}
\newcommand{\bottomtitlebar}{
  \vskip 0.29in
  \vskip -\parskip
  \hrule height 1pt
  \vskip 0.09in%
}
\renewcommand{\maketitle}{
  \vbox{%
%    \hsize\textwidth
%    \linewidth\hsize
%    \vskip 0.1in
    \toptitlebar
    \centering
    {\Large\bf 6.867: Homework 3}
    \bottomtitlebar
  }
}

\newcommand{\brows}[1]{%
  \begin{bmatrix}
  \begin{array}{@{\protect\rotvert\;}c@{\;\protect\rotvert}}
  #1
  \end{array}
  \end{bmatrix}
}
\newcommand{\rotvert}{\rotatebox[origin=c]{90}{$\vert$}}
\newcommand{\rowsvdots}{\multicolumn{1}{@{}c@{}}{\vdots}}


\setlength{\parindent}{0pt}
\setlength{\parskip}{5.5pt}

% table
\usepackage{multirow}
\newcommand{\ra}[1]{\renewcommand{\arraystretch}{#1}}
\usepackage{float}

% To draw figures
\usepackage{graphicx}
\usepackage{caption,subcaption}
\usepackage{mathtools}

\begin{document}
\maketitle

\section{Globally Linear Embedding}
\subsection{Show that $M=\left(I_{m}-W_{n}\right)^{T}\left(I_{m}-W_{n}\right)$ is a positive semi-definite matrix}
\vspace*{1mm}
\noindent To show a $m \times m$ matrix $M$ is positive semidefinite, we need to show follows:
\begin{align*}
  &\text{1. M is a symmetric matrix}\\
  &\text{2. $\x^TM\x \ge 0$ for all $\x \in \mathbb{R}^m$ }
\end{align*}

\noindent For 1, symmetric property, we can show this property as follows:
\begin{align*}
  M^T &= (\left(I_{m}-W_{n}\right)^{T}\left(I_{m}-W_{n}\right))^T\\
  &= \left(I_{m}-W_{n}\right)^T(\left(I_{m}-W_{n}\right)^T)^T\\
  &= \left(I_{m}-W_{n}\right)^{T}\left(I_{m}-W_{n}\right)\\
  &= M\\
  & \iff \text{M is symmetric matrix}\\
\end{align*}

\noindent For 2, all nonnegative entries condition, we can show this property as follows:
\begin{align*}
  &\text{Let's denote } I_{m}-W_{n} \text{ as } A. \text{ Then } M=A^{T}A.\\
  &\text{For given column vector } \x, \\
  & \x^TM\x = \x^{T}A^{T}A\x = (A\x)^{T}A\x \ge 0\\
  \tag{QED}&\iff M \text{ is positive semidefinite.}
\end{align*}

\subsection{Re-express the objective function $\Phi(Y)$}
\vspace*{1mm}
$\Phi(Y)$ is sum of errors how much $y_i$, a lower dimensional representation of $x_i$, deviates from its reconstruction $\sum_{j=1}^{m}\left(W_{n}\right)_{i, j} y_{j}$, a weighted linear combination of its $n$-nearest neigbours.
As we can check from the below definition, the errors are measured by square of L-2 norm.
%Let's consider a matrix $Y-\Wn Y \in \mathbb{R}^{m \times d}$
$$\Phi(Y):=\sum_{i=1}^{m}\left\|y_{i}-\sum_{j=1}^{m}\left(W_{n}\right)_{i, j} y_{j}\right\|^{2}$$

\noindent Let's express such errors, which measured per each data points, as $\epsilon_i \triangleq y_{i} - \sum_{j=1}^{m}\left(W_{n}\right)_{i, j} y_{j} \text{ for } i \in \{1, \hdots, m\}$.
Then we can notice $\epsilon_i \in \mathbb{R}^d$. In addition, if we notate $l$-th dimension of $\epsilon_i$ as $\epsilon_i^l$, we can understand that squre of L-2 norm of an error of $i$-th data point as $\left\|y_{i}-\sum_{j=1}^{m}\left(W_{n}\right)_{i, j} y_{j}\right\|^{2}=||\epsilon_i||^2 = \sum_{l=1}^d (\epsilon_i^l)^2$.
So using the \textit{\textbf{Hint}}, we can aggregate all $(\epsilon_i^l)^2$ for all $i \in \{1, \hdots, m\}$ and $l \in \{1, \hdots, d\}$ by defining $A$ in Frobenius norm $\|A\|_{\mathrm{F}}=\sqrt{\operatorname{Tr}\left(A^{T} A\right)}$ as follows:
$$
A \triangleq \brows{\epsilon_1^T \\ \epsilon_2^T \\ \rowsvdots \\ \epsilon_m^T}
$$
Then
\begin{align*}
  \|A\|_{\mathrm{F}}^{2}&=\operatorname{Tr}\left(A^{T} A\right)\\
  &=\operatorname{Tr}\left(\begin{bmatrix}
    \vert &  \vert & &\vert \\
    \epsilon_1 & \epsilon_2 & \hdots &\epsilon_m \\
    \vert & \vert& & \vert
\end{bmatrix}
\brows{\epsilon_1^T \\ \epsilon_2^T \\ \rowsvdots \\ \epsilon_m^T}\right)\\
  &=\operatorname{Tr}\left(
  \begin{bmatrix}
    \sum_{i=1}^m(\epsilon_i^1)^2 &&& \\
    &\sum_{i=1}^m(\epsilon_i^2)^2 &&&\\
    &&\ddots && \\
    &&&    \sum_{i=1}^m(\epsilon_i^d)^2& \\
  \end{bmatrix}\right)\\
  &=\sum_{i=1}^m\sum_{l=1}^d(\epsilon_i^l)^2\\
  &=\sum_{i=1}^m\left\|y_{i}-\sum_{j=1}^{m}\left(W_{n}\right)_{i, j} y_{j}\right\|\\
  \tag{1}&=\Phi(Y)
\end{align*}

\noindent Therefore, since $\epsilon_i \triangleq y_{i} - \sum_{j=1}^{m}\left(W_{n}\right)_{i, j} y_{j}$ and $A$ is row-wise stack up of $\epsilon_i$s, we can represent $A$ using $Y$ and $\Wn$ as follows:
\begin{align*}
A &\triangleq \brows{\epsilon_1^T \\ \epsilon_2^T \\ \rowsvdots \\ \epsilon_m^T} \\
  \tag{2}&=Y-\Wn Y
\end{align*}
Thus, $\Phi(Y) =\|Y- \Wn Y\|_{\mathrm{F}}^{2}$ by using (1) and (2).

\subsection{Write the Lagrangian of the problem using Part 2}
\vspace*{1mm}
Re-writing Lagrangian is started from incorporating the constraints into the objective. As guided in the problem, we only incorporates the first constraint, $\frac{1}{m} \sum_{i=1}^{m} y_{i} y_{i}^{T}=I_{d}$. Then the Lagrangian $L(Y)$ is written as follows:
$$
L(Y) = \sum_{i=1}^{m}\left\|y_{i}-\sum_{j=1}^{m}\left(W_{n}\right)_{i, j} y_{j}\right\|^{2} - \sum_{\alpha, \beta =1}^d \lambda_{\alpha\beta}\left[\frac{1}{m}\sum_{i=1}^my_i^\alpha y_i^\beta - \delta_{\alpha\beta}\right]
$$
where $\lambda_{\alpha\beta}$ represents Lagrange multiplier(s) which satisfies $\lambda_{\alpha\beta} = \lambda_{\beta\alpha}$, $y_i^{\alpha}$ and $y_i^{\beta}$ represents $\alpha$-th or $\beta$-th dimensional element of $y_i$, and $\delta_{\alpha\beta}$ represents Kronecker delta, with $\delta_{\alpha \beta}=0 \text { if } \alpha \neq \beta \text { and } \delta_{\alpha \beta}=1 \text { if } \alpha = \beta$.
Intuitively, the second term with Lagrange multiplier $\lambda_{\alpha\beta}$ enforces $\frac{1}{m} \sum_{i=1}^{m} y_{i} y_{i}^{T}$ to be $d \times d$ identity matrix, $I_{d}$. \\
Now, using the conclusion from the Part2, let's re-write the first term of the Lagrangian, $\sum_{i=1}^{m}\left\|y_{i}-\sum_{j=1}^{m}\left(W_{n}\right)_{i, j} y_{j}\right\|^{2}$.

\begin{align*}
\comment{conclusion of Part 2.}\sum_{i=1}^{m}\left\|y_{i}-\sum_{j=1}^{m}\left(W_{n}\right)_{i, j} y_{j}\right\|^{2} &= ||Y-\Wn Y||_F^{2}\\
&= ||(I_m-\Wn)Y||_F^{2}\\
\comment{$||A||_F = \sqrt{\operatorname{Tr}(A^TA)}$}&= \operatorname{Tr}((I_m-\Wn)Y)^T(I_m-\Wn)Y))\\
&= \operatorname{Tr}(Y^T(I_m-\Wn)^T(I_m-\Wn)Y) \\
\comment{$M=\left(I_{m}-W_{n}\right)^{T}\left(I_{m}-W_{n}\right)$}&= \operatorname{Tr}(Y^TMY)
\end{align*}\

\noindent Thus, we get the first term of our goal. Now we can also re-write the second term $\sum_{\alpha, \beta =1}^d \lambda_{\alpha\beta}\left[\frac{1}{m}\sum_{i=1}^my_i^\alpha y_i^\beta - \delta_{\alpha\beta}\right]$ as follows using $\operatorname{Tr}$:
$$\operatorname{Tr}\left(\Lambda\left(\frac{1}{m} Y^{T} Y-I_{d}\right)\right)$$
where $\Lambda=\left[\lambda_{\alpha \beta}\right]_{1 \leq \alpha, \beta \leq d} \in \mathbb{R}^{d \times d}$. From the above expression, we can understand $\operatorname{Tr}$ aggregates all element-wise errors which is correspondent to two summations, $\sum_{\alpha, \beta =1}^d$ and $\sum_{i=1}^my_i^\alpha y_i^\beta$, and $\Lambda$ assorts all Lagrangian multipliers in matrix form.\\
Now, let's re-write the entire Lagrangian $L(Y)$ using the first and second term as follows:

\begin{align*}
\tag{1} L(Y)&=\operatorname{Tr}\left(Y^{T} M Y\right)-\operatorname{Tr}\left(\Lambda\left(\frac{1}{m} Y^{T} Y-I_{d}\right)\right)\\
  &= ||(I_m-\Wn)Y||_F^{2} - \operatorname{Tr}\left(\Lambda\left(\frac{1}{m} Y^{T} Y-I_{d}\right)\right)
\end{align*}

%\noindent One remaining difference from our final goal and the above formulation is that $\Lambda$ is not \textit{diagonal} in the above case. To fix this, let's think about an orthogonal matrix $U \in O(d)$ s.t. $U^T\Lambda U=\tilde{\Lambda}$ where $O(d)$ defines the set including all $d \times d$ real orthogonal matrices and $\tilde{A}$ is a diagonal matrix.
%We're going to show diagonal form of $\Lambda$ actually has the same solution with the entire entries' form of $\Lambda$ by solving the above Lagrangian.

To solve the Lagrangian, we take partial deriviate of the above Lagrangian $L(Y)$ in terms of Y as follows:

\begin{align*}
  2 \cdot \frac{\delta \mathscr{L(Y)}}{\delta Y}&= \left(I_{m}-\Wn\right)^T\left(I_{m}-\Wn\right)Y-\frac{1}{m} Y \Lambda^T \\
  \tag{2} \comment{$\lambda_{\alpha\beta} = \lambda_{\beta\alpha}$, thus $\Lambda$ is symmetric matrix}&= \left(I_{m}-\Wn\right)^T\left(I_{m}-\Wn\right)Y-\frac{1}{m} Y \Lambda\\
  \intertext{Since symmetric matrix is diagonalizable, so let $\Lambda\triangleq U\tilde{A}U^T$ where $U \text { a } d \times d \text { orthogonal matrix and } \tilde{\Lambda}=\operatorname{diag}\left(d_{1}, \ldots, d_{d}\right)$}
    &= \left(I_{m}-\Wn\right)^T\left(I_{m}-\Wn\right)Y-\frac{1}{m} Y U\tilde{\Lambda}U^T
\end{align*}
If we multiply $U$ to both sides and let $\tilde{Y}$ \triangleq YU$, then
\begin{align*}
  2 \cdot \frac{\delta \mathscr{L(Y)}}{\delta Y}U&= \left(I_{m}-\Wn\right)^T\left(I_{m}-\Wn\right)YU-\frac{1}{m} Y U\tilde{\Lambda}U^{T}U\\
  \intertext{Using the property of orthgonal matrix $U^TU = I_d$}
  \tag{3} &= \left(I_{m}-\Wn\right)^T\left(I_{m}-\Wn\right)\tilde{Y} -\frac{1}{m} \tilde{Y}\tilde{\Lambda}
\end{align*}

By comparing (2) and (3), we can understand that the original Lagrangian's FOC condition is transformed into the case where the matrix containing Lagrangian multipliers is \textbf{diagonal} under the orthogonal transformation $U$.
This means that for given any solution $(Y^*, \Lambda^*)$ of the original Lagrangian, we can derive another form of solution $(\tilde{Y}^*, \tilde{\Lambda}^*)$ where the matrix of Lagrangian multipliers is diagonal using the orthgonal transformation $U$.

Therefore, w.l.o.g., we can change the formulation of original Lagrangian (1) with a diagonal $\Lambda \triangleq \operatorname{diag}\left(d_{1}, \ldots, d_{d}\right)$ as follows:
%$Y}$ and $\tilde{\Lambda}$ into $Y$ and $\Lambda$ to go back to the notation given by the problem set and it solves the Part 3 because both Lagrangians $L(Y)$ and $L_U(\tilde{Y})$ are equivalent statements in terms of optimization.
\begin{align*}
\tag{4} L(Y)=\operatorname{Tr}\left(Y^{T} M Y\right)-\operatorname{Tr}\left(\Lambda\left(\frac{1}{m} Y^{T} Y-I_{d}\right)\right)
\end{align*}
%
%\noindent Since it's known that Lagrangian is invariant under such orthogonal transformation, we can re-write our Lagrangian as follows by :
%
%$$L_U(\tilde{Y})=\operatorname{Tr}\left(\tilde{Y}^{T} M \tilde{Y}\right)-\operatorname{Tr}\left(\tilde{\Lambda}\left(\frac{1}{m} \tilde{Y}^{T}\tilde{Y}-I_{d}\right)\right)$$
%\noindent where $\tilde{Y}=U^{T}Y$ and $\tilde{\Lambda}=U^T\Lambda U$. We can check that this formulation incorporates the first constraint as well from the formulation as follows:
%
%$$
%\frac{1}{m} \tilde{Y}^{T}\tilde{Y}=U^{T}YY^TU = UI_dU^T = UU^T = I_d
%$$
%
%\noindent In this formulation, $\tilde{A}$ is a diagonal matrix by construction. Therefore, w.l.o.g., let's re-write $\tilde{Y}$ and $\tilde{\Lambda}$ into $Y$ and $\Lambda$ to go back to the notation given by the problem set and it solves the Part 3 because both Lagrangians $L(Y)$ and $L_U(\tilde{Y})$ are equivalent statements in terms of optimization.
%$$L(Y)=\operatorname{Tr}\left(Y^{T} M Y\right)-\operatorname{Tr}\left(\Lambda\left(\frac{1}{m} Y^{T} Y-I_{d}\right)\right)$$

%If we denote a collection of solution of Lagrangian and its corresponding multiplier as $\varphi \triangleq \left[Y^*, \Lambda^* \right]$.
%It is known that such solution $\varphi$ invariant to the transformation under $U$, i.e., $\varphi_U = [U^TY^*, U^T\Lambda^*U]$. (This is because such transformation appears as the dot product inside the formulation of the original and transformed Lagrangian, $L$ and $L_G$, and orthogonal transformation preserves dot product.)

%$L(Y)=\operatorname{Tr}\left(Y^{T} M Y\right)-\operatorname{Tr}\left(\Lambda\left(\frac{1}{m} Y^{T} Y-I_{d}\right)\right)$

\subsection{Solving the constrained LLE optimization problem}
\vspace*{1mm}
%Let's notate the solution of Part 3's Lagrangian (4) as $(Y^*, \Lambda^*)$. Then such solution is transformed into $(Y^*P, P\Lambda^*P^T)= (Y^*P, \Lambda^*_{asc})$ where $P$ is the permutation matrix having the property as follows:
%To re-arrange the above defined $\Lambda \triangleq \operatorname{diag}\left(d_{1}, \ldots, d_{d}\right)$ in ascending order, let's define a permutation matrix $P \in \mathbb{R}^{d \times d}$ as follows:
%$$
%P\Lambda P^T=\Lambda'=\operatorname{diag}\left(d_{1}^{\prime}, \ldots, d_{d}^{\prime}\right) \text { with } d_{1}^{\prime} \leq \cdots \leq d_{d}^{\prime}
%$$
%It's worth noting that permutation matrix $P$ is orthogonal because $PP^T=P^TP=I_d$ because applying same permutation twice in a row doesn't make any change.

Now, let's solve the FOC condition of the Lagrangian from (2):
\begin{align*}
&\left(I_{m}-\Wn\right)^T\left(I_{m} - \Wn\right)Y-\frac{1}{m} Y \Lambda =0 \\
&\iff \left(I_{m}-\Wn\right)^T\left(I_{m} - \Wn\right)Y = \frac{1}{m} Y \Lambda
%\\&\iff \left(I_{m}-\Wn\right)^T\left(I_{m} - \Wn\right)YP = \frac{1}{m} YP \Lambda \\
%&\iff \left(I_{m}-\Wn\right)^T\left(I_{m} - \Wn\right)YPP^T = \frac{1}{m} YP \Lambda P^T\\
%\comment{$PP^T = I_d$}&\iff \left(I_{m}-\Wn\right)^T\left(I_{m} - \Wn\right)Y = \frac{1}{m} Y\Lambda'\\
\end{align*}
If we look at the above formulation column-wisely, it corresponds to the definition of eigenvector and eigenvalue as follows:
\begin{align*}
&\left(I_m- \Wn \right)^T\left(I_m- \Wn\right) v_{i}=\lambda_{i} v_{i}, \quad \lambda_{i}=\frac{1}{m} d_{i} \text{ for } i = \{1, 2, \hdots, d\} \text{ where }  v_i \text{ is } i\text{-th column of } Y \text{ and } d_i \text{ is } i\text{-th diagonal element of } \Lambda. \\
&\iff Mv_i =\lambda_{i} v_{i}, \quad \lambda_{i}=\frac{1}{m} d_{i} \text{ for } i = \{1, 2, \hdots, d\}
\end{align*}
Now from the property found in Part 1, since $M$ is positive semidefinite, $\lambda_{i}$s are all non-negative.
Moreover, $[1, 1, \hdots, ... , 1]^T$ which is a column vector with $m$ number of $1$s is $\alpha_1$ which is an eigenvector associated with smallest eigenvalue $0$. This is because $(I_m-\Wn)\alpha_1 = \mathbf{0}$ from the given two conditions - i) $\left(W_{n}\right)_{i j}=0 \text { for } x_{j} \notin \mathcal{N}_{n}\left(x_{i}\right)$ and ii) $\sum_{j=1}^{m}\left(W_{n}\right)_{i j}=1 \text { for } i \in\{1, \ldots, m\}$ - where row-wise sum using $\alpha_1$ of $I_{m}-\mathrm{W}_{\mathrm{n}}$ is always set to be $0$ for each row ($1$ from $I_m$ is subtracted with $1$ from the summation of $i \ne j$ entries for each row).
%row of $(W_n)$ includes elements sum up to $1$ except the element at diagonal positions.
This fact implies that $\alpha_1$ and $0$ is an eigenvector and corresponding eigenvalue for $\left(I_m- \Wn \right)^T\left(I_m- \Wn\right)$, i.e. $\left(I_m- \Wn \right)^T\left(I_m- \Wn\right)\alpha_1 = 0\cdot\alpha_1$.

In addition, by left multiplying $Y^T$ on the both sides of $\left(I_{m}-\Wn\right)^T\left(I_{m} - \Wn\right)Y = \frac{1}{m} Y\Lambda$ we get as follows:
\begin{align*}
&Y^T\left(I_{m}-\Wn\right)^T\left(I_{m} - \Wn\right)Y = \frac{1}{m} Y^TY\Lambda\\$
&\iff Y^TMY= \Lambda \quad \text{(using $M=\left(I_{m}-\Wn\right)^T\left(I_{m} - \Wn\right)$ and $\frac{1}{m}Y^TY=I_d$) }
\end{align*}

Since we are minimizing $\Phi(Y) =\|Y- \Wn Y\|_{\mathrm{F}}^{2} = ||(I_m - \Wn)Y||_F^2=\operatorname{Tr}(Y^T(I_m - \Wn)^T(I_m - \Wn)Y) = \operatorname{Tr}(Y^TMY) = \operatorname{Tr}(\Lambda) =  \sum_{i=1}^{d} d_i =  m\sum_{i=1}^d\lambda_i$, we can minimize this $\Phi(Y)$ by choosing the smallest $d$ number of eigenvalues' corresponding eigenvectors as $v_i$s except the case of eigenvalue of $0$ and its correspondent eigenvector $\alpha_1 = [1, 1, \hdots, 1]^T$. This is because $\alpha_1$ deviates from the condition that $\sum_{i=1}^{m} y_{i} = \mathbf{0}$ because $\sum_{i=1}^{m} (\alpha_1)_m = m  >0$. Therefore, since $\lambda_i$s are all non-negative, if we line up all eigvenvalues from $0$ to ascending order, then $d$ number of corresponding eigenvectors except $\alpha_1$ will be $\alpha_2, \alpha_3, \hdots, \alpha_{d+1}$, upto a scaling constant, as the Part 4 expects. \tag{QED}

%Then let's multiply the FOC condition of our Lagrangian (2)
%Since permutation matrix $P$ is orthogonal because $PP^T=P^TP$ should be $I_d$,

\section{Learning Hidden Markov Models}
\subsection{Derive $\alpha(z_{t}=k)$}

Let's start from the state where $z_{t-1} = j$. Then we can get to the next state of $z_{t} = k$ with the probability of an entry of transition matrix $A$, $A_{jk}$.
Then by summing up such quantity over all possible $j$s, we get as follows:

$$
\sum_{j=1}^k \alpha(z_{t-1}=j)\cdot A_{jk}
$$

Then by multiply next step that is $z_{t} = k$ omitting $y_{t}$ with probability of $p(y_{t}|z_{t}=k)$, we get what we want:

$$
\alpha(z_{t}=k) = p(y_{t}|z_{t}=k)\sum_{j=1}^k \alpha(z_{t-1}=j)\cdot A_{jk}
$$

\subsection{Derive $\gamma(z_{t}=k)$}
\begin{align*}
\gamma(z_{t}=k) &= p(z_t =k | Y)\\
&= \frac{p(z_t=k , Y)}{p(Y)}\\
&= \frac{p(y_{t+1}, y_{t+2}, \hdots, y_T|y_1, y_2, \hdots, y_t, z_t = k)\cdot p(y_1, y_2, \hdots, y_t, z_t = k)}{p(Y)}\\
&= \frac{p(y_{t+1}, y_{t+2}, \hdots, y_T|y_1, y_2, \hdots, y_t, z_t = k)\cdot \alpha(z_t = k)}{p(Y)}\\
\comment{Conditional Independence} &= \frac{p(y_{t+1}, y_{t+2}, \hdots, y_T|z_t = k)\cdot \alpha(z_t = k)}{p(Y)}\\
&= \frac{\beta(z_t = k)\cdot \alpha(z_t = k)}{p(Y)}\\
\end{align*}

\subsection{Derive $\pi_{k}^*$}
Let's start from constructing Lagrangian $\mathcal{L}$ with a constraint on $\pi$, $\sum_{k=1}^{K} \pi_{k}=1$ as follows:
\begin{align*}
\mathcal{L} &= Q\left(\Theta^{\text {new }}, \Theta\right) - \lambda \cdot (\sum_{k=1}^{K} \pi_{k} - 1)\\
&= \sum_{k=1}^{K} \gamma\left(z_{1}=k\right) \ln \left(\pi_{k}\right) +\sum_{t=2}^{T} \sum_{j=1}^{K} \sum_{k=1}^{K} \xi\left(z_{t-1}=j, z_{t}=k\right) \ln \left(A_{j k}\right) +\sum_{t=1}^{T} \sum_{k=1}^{K} \gamma\left(z_{t}=k\right) \ln \left(p\left(y_{t} \mid z_{t}=k\right)\right) - \lambda \cdot (\sum_{k=1}^{K} \pi_{k} - 1)
\end{align*}

Now take the partial derivative of $\mathcal{L}$ over $\pi_k$ gives:
\begin{align*}
\frac{\partial \mathcal{L}}{\partial \pi_k} &= \frac{\gamma\left(z_{1}=k\right)}{\pi_k} - \lambda\\
&= 0\\
&\iff \lambda = \frac{\gamma\left(z_{1}=k\right)}{\pi_k}
\end{align*}

Now using $\lambda=\frac{\gamma\left(z_{1}=k\right)}{\pi_k}$ and $\sum_{k=1}^{K} \pi_{k} = 1$ we get:
\begin{align*}
\sum_{k=1}^{K} \pi_{k} = \frac{1}{\lambda}\sum_{k=1}^K \gamma(z_1 = k) = 1 &\iff \lambda = \sum_{k=1}^K \gamma(z_1 = k) \\
&\Rightarrow \pi_k^* = \frac{\gamma(z_1 = k)}{\sum_{k=1}^K \gamma(z_1 = k)}\\
\comment{$\sum_{k=1}^K \gamma(z_1 = k)= 1$} & \iff \pi_k^* = \gamma(z_1 = k) \quad \forall 1 \le k \le K
\end{align*}

\subsection{Derive $\mu_{k}^*$ and  $\sigma_{k}^{2*}$}
Let's start from $p(y_t | z_{t}=k) \sim \mathcal{N}(\mu_k, \sigma_k^2)$ whose PDF is written as follows:
\begin{align*}
& p(y_t | z_{t}=k) = \frac{1}{\sigma_k \sqrt{2 \pi}} e^{-\frac{1}{2}\left(\frac{y_t-\mu_k}{\sigma_k}\right)^{2}} \\
&\iff \ln(p(y_t | z_{t}=k)) =  -\ln\sigma_k - \frac{1}{2\sigma_k^2}(y_t - \mu_k)^2 - \ln(\sqrt{2\pi})
\end{align*}
Recalling that $\gamma \text { and } \xi$ are regarded as constant in $M$-step, the only term that contains $p(y_t | z_{t}=k)$ is the third term of $Q\left(\Theta^{\text {new }}, \Theta\right)$.
So now let's compute partial derivative of $Q\left(\Theta^{\text {new }}, \Theta\right)$ w.r.t. $\mu_k$:
\begin{align*}
\frac{\partial Q\left(\Theta^{\text {new }}, \Theta\right)}{\partial \mu_k} &=  \sum_{t=1}^{T}\gamma(z_t=k)\frac{1}{\sigma_k^2}(y_k-\mu_k) = 0\\
&\Rightarrow \mu_{k}^*=\frac{\sum_{t=1}^T \gamma\left(z_{t}=k\right) y_{t}}{\sum_{t=1}^T \gamma\left(z_{t}=k\right)}
\end{align*}

Similarly,

\begin{align*}
\frac{\partial Q\left(\Theta^{\text {new }}, \Theta\right)}{\partial \sigma_k^2} &= \sum_{t=1}^T \gamma\left(z_{t}=k\right) \left[ - \frac{1}{\sigma_k} + \frac{1}{\sigma_k^3}(y_t - \mu_k)^2 \right] = 0\\
&\iff \sum_{t=1}^T \gamma\left(z_{t}=k\right)[-\sigma_k^2 + (y_t - \mu_k)^2] = 0\\
&\Rightarrow \sigma_{k}^2^*=\frac{\sum_{t=1}^T \gamma\left(z_{t}=k\right) (y_t - \mu_k)^2}{\sum_{t=1}^T \gamma\left(z_{t}=k\right)}
\end{align*}

\subsection{Upper triangular $A$}
Upper triangular $A$ should satisfy following quantity:

$$
A_{jk} \begin{align} = 0 \quad \text{if} \quad j > k  \end{align} \quad \text{and} \quad \sum_{j=1}^K A_{jk}} = 1
$$

In addition, $A_{jk}$ is represented as follows:

$$
A_{j k}=\frac{\sum_{t=2}^{T} \xi\left(z_{t-1}=j, z_{t}=k\right)}{\sum_{l=1}^{K} \sum_{t=2}^{T} \xi\left(z_{t-1}=j, z_{t}=l\right)}
$$

Therefore, in case $j > k$, $\xi\left(z_{t-1}=j, z_{t}=k\right)=0 \quad \forall t=\{2, 3, \hdots, T\}$ to ensure $A$ to be upper triangular.\\

For $\xi\left(z_{t-1}=j, z_{t}=k\right)=\frac{\alpha\left(z_{t-1}=j\right) p\left(y_{t} \mid z_{t}\right) A_{jk} \beta\left(z_{t}=k\right)}{p(\mathbf{Y})}$ to be $0$, $A_{jk}$ must be $0$. In other words, \textbf{in case of $j > k$, $A_{jk}$ must be initialized as $0$ to ensure A to be upper traingular} regardless of with which quantity any other parameter - $\alpha, \beta, p(y|z), p(Y)$ - being assigned at any step of iterations.


For the case of $j \le k$, $A_{jk}$ can be initialized as any quantity if they satisfy $\sum_{k=j}^K A_{jk}} = 1$. To do so, we can leverage Dirichlet distribution that can enforce the sum of entries to be $1$. Since we already have $j-1$ number of $0$s for each $j-th$ row from the case of $j > k$, $K-j+1$ number of $j \le k$ entries should be sum up to $1$.
To do so, \textbf{we can initialize $A_{jk}$s for $j \le k$ as follows}:

$$A_{j,j}, A_{j,j+1}, \hdots, A_{j,K} \sim \operatorname{Dir}(\alpha) \text{ where } \alpha = [1, 1, \hdots, 1] \in \mathbb{R}^{K-j+1}\quad \forall j \in \{1, 2, \hdots, K\}$$

This will enforce the upper triangular entries to be sum up to $1$.

\end{document}
